\section{Introduction}

\section{Theory}
\cite{hofmann2015}

\subsection{Crystal Lattice Structure}

To calculate the distance between planes the formula 
\begin{equation}
\frac{1}{d^2}=\frac{h^2}{a^2}+\frac{k^2}{b^2}+\frac{l^2}{c^2}.
\label{eq:seperation}
\end{equation}
If we use this formula for a simple cubic with $a=b=c=\SI{0.432}{\nano\m}$ for the plane(111) we get the separation to $d=\SI{0.249}{\nano\m}$. If we do the same for the plane (211) we obtain a separation of $d=\SI{0.176}{\nano\m}$ and for the plane (100) we get $d=\SI{0.432}{\nano\m}$. 



\subsection{Diffraction}

To evaluate diffraction patterns Bragg's law is mainly used, with is defined as
\begin{equation}
    n\lambda=2d\sin \theta,
    \label{eq:Bragg}
\end{equation}
where $n$ is an integer, $\lambda$ is the wavelength of the incoming photon, $d$ is the spacing between the planes and $\theta$ is the scattering angle. Using this we see that if the incoming x-ray has a wavelength of \SI{0.07}{\nano\m} we can at minimum spacing which can be observed is \SI{0.35}{\angstrom}. In order to observe smaller spacings, a higher photon energy would have to be used. 

For example if we have a (111) plane with an  incident angle \( \theta = 11.2^\circ \) when X-rays of wavelength of \( 0.154 \, \text{nm} \) then the side of the side of each unit cell can be calculated as follows: From \autoref{eq:Bragg} we obtain a the distance planer separation to $d=\SI{0.397}{\nano\m}$, from \autoref{eq:seperation} we obtain that a separation of $d=\SI{0.397}{\nano\m}$ leads to a lattice constant of $a=\SI{6.36}{\angstrom}$ for a simple cubic. 

Another example for a orthorhombic unit cell with dimensions $a=\SI{5.74}{\angstrom}$, $b=\SI{7.96}{\angstrom}$ and $c=\SI{4.95}{\angstrom}$ one can calculate the incident angles for the (100), (010), and
(111) planes if the  wavelength is \SI{0.083}{\nano\m}. For the (100) plane \autoref{eq:seperation} gives a planar separation of $d_{(100)}=\SI{0.574}{\nano\m}$, (010) gives $d_{(010)}=\SI{0.796}{\nano\m}$ and (111) gives $d_{(111)}=\SI{0.339}{\nano\m}$. Using \autoref{eq:Bragg} the following angles are found: $\theta_{(100)} = \SI{4.15}{\degree}$, $\theta_{(010)} = \SI{2.99}{\degree}$ and $\theta_{(111)} = \SI{7.03}{\degree}$.

\subsection{Structure Factors}
Due to the structure of different types of lattices, the diffraction pattern may change. This is due to structure function which are defined as 
\begin{equation}
    F(hkl) = \sum_{m} f_m \exp\left( 2\pi i (u_m h + v_m k + w_m l) \right). 
    \label{eq:structurefunction}
\end{equation}
where $f_m$ is the atomic form factor and $(u_m,v_m,w_m)$ are the fractional coordinates of atoms in the unit cell. for a bcc we know that we have the fractional coordinates (0,0,0) for each corner and (1/2,1/2,1/2). and for an fcc we have the (0,0,0) for each corner,  (0,1/2,1/2), (1/2,0,1/2) and (1/2,1/2,0). In the equation above h,k,l is the value of the miller index. Putting all of this together we know that we will obtain diffraction peaks only if $F(hkl)\neq0$ \cite{solidstatephysics2025}. 

From this we can evaluate which planes we wil lsee from bcc and fcc. 



\subsection{Powder X-ray Diffraction}

Transmission, fluorescence, 

Xray diffraction, oscillating E field. When X ray hit the electrons the electrons start o vibrate in that frequency
- Constructive and destructive interference 

Bragg's law: $n\lambda=PD=2s=2d\sin\theta$

Crystal lattice: fcc, bcc


In powder diffraction we have all possible orientation sof the crystallites and some will be oriented in the right way for diffraction. Give rise to powder diffraction rings. 

Scherrer's formula
\begin{equation}
    t=\frac{k\lambda}{\beta\cos\theta}
\end{equation}

Structure factor: Se photos. For a simple cubic we see all peaks, for bcc and fcc we loose some peaks. 

