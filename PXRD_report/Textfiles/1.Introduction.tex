\section{Introduction}

\section{Theory}
\cite{hofmann2015}

\subsection{Crystal Lattice Structure}

To calculate the distance between planes the formula 
\begin{equation}
\frac{1}{d^2}=\frac{h^2}{a^2}+\frac{k^2}{b^2}+\frac{l^2}{c^2}.
\end{equation}
If we use this formula for a simple cubic with $a=b=c=\SI{0.432}{\nano\m}$ for the plane(111) we get the separation to $d=\SI{0.249}{\nano\m}$. If we do the same for the plane (211) we obtain a separation of $d=\SI{0.176}{\nano\m}$ and for the plane (100) we get $d=\SI{0.432}{\nano\m}$. 



\subsection{Diffraction}

\subsection{Powder X-ray Diffraction}

Transmission, fluorescence, 

Xray diffraction, oscillating E field. When X ray hit the electrons the electrons start o vibrate in that frequency
- Constructive and destructive interference 

Bragg's law: $n\lambda=PD=2s=2d\sin\theta$

Crystal lattice: fcc, bcc

\begin{equation}
    n\lambda=2d \sin \theta
    \label{eq:Bragg}
\end{equation}

In powder diffraction we have all possible orientation sof the crystallites and some will be oriented in the right way for diffraction. Give rise to powder diffraction rings. 

Scherrer's formula
\begin{equation}
    t=\frac{k\lambda}{\beta\cos\theta}
\end{equation}

Structure factor: Se photos. For a simple cubic we see all peaks, for bcc and fcc we loose some peaks. 

