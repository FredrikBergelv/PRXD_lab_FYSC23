\section{Introduction}

\section{Theory}
\cite{hofmann2015}

\subsection{Crystal Lattice Structure}

\subsection{Diffraction}

\subsection{Powder X-ray diffraction}

Transmission, fluorescence, 

Xray diffraction, oscillating E field. When X ray hit the electrons the electrons start o vibrate in that frequency
- Constructive and destructive interference 

Bragg's law: $n\lambda=PD=2s=2d\sin\theta$

Crystal lattice: fcc, bcc

In powder diffraction we have all possible orientation sof the crystallites and some will be oriented in the right way for diffraction. Give rise to powder diffraction rings. 

Scherrer's formula
\begin{equation}
    t=\frac{k\lambda}{\beta\cos\theta}
\end{equation}

Structure factor: Se photos. For a simple cubic we see all peaks, for bcc and fcc we loose some peaks. 

